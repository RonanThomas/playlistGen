\documentclass[a4paper,titlepage,oneside]{article}

\usepackage{DocumentationUtilisateur}
\begin{document}
\pagestyle{empty}

%%% Première couverture %%%
\imageWOCaption{data/logo_pg.png}{width=5cm}{logo}
\begin{center}
\vspace{3.5cm}
\Huge{\textbf{Générateur de playlist}}\\
\Huge{PlaylistGen}\\
\vspace{1cm}
\huge{Documentation utilisateur}
\vfill
\end{center}
\begin{flushleft}
Ronan \bsc{Legardinier}\\
Thomas \bsc{Dabre}\\
\hfill
\today
\end{flushleft}
\newpage
%%% Première couverture %%%

\newpage
\pagestyle{fancy}
\setcounter{page}{1}
\HEADER{Table des matières}
\tableofcontents{}
\newpage


\HEADER{Documentation utilisateur}
\section{Configuration}
La configuration de l'application passe par le script \texttt{configure.py}.\\

Pour configurer l'application, \texttt{configure.py} requière :
\begin{itemize}
	\item L'identifiant de la base de données.
	\item Le mot de passe la base de données.
	\item L'adresse IP ou l'URI hôte de la base de données.
	\item Le port d'accès à la base de données.
\end{itemize}

\subsection{Exemple}
\image{data/configure.png}{width=\linewidth}{Screenshot : configure.py}{fig:configure}

\section{Utilisation}
L'application se lance ensuite avec le script \texttt{main.py}.\\

L'utilisation de l'application requière trois arguments obligatoires tels que :
\begin{itemize}
	\item La durée totale de la playlist (en minutes).
	\item Le nom du fichier de sortie de la playlist.
	\item Le format de la playlist (\textit{M3U, XSPF, PLS}).\\
\end{itemize}

\textboxed{\linewidth}{REMARQUE}{L'utilisation de ces trois arguments seuls génère une playlist totalement aléatoire.}\\\\

L'application peut également recevoir des arguments facultatifs tels que :
\begin{center}
	\begin{tabularx}{8cm}{|X|l|}
		\hline
		\textbf{Argument} & \textbf{Écriture}\\
		\hline
		\hline
		Genre		& -g unGenre unPourcentage\\
		Sous-genre	& -G unSousGenre unPourcentage\\
		Artiste		& -a unArtiste unPourcentage\\
		Album		& -A unAlbum unPourcentage\\
		Titre		& -t unTitre unPourcentage\\
		\hline
	\end{tabularx}
\end{center}

\newpage
\subsection{Exemples}
\image{data/main.png}{width=\linewidth}{Screenshot : main.py}{fig:main}

Dans la premier exemple, la playlist générée au format m3u, d'une durée totale de 60 minutes, sera totalement aléatoire.

Tandis que dans le second exemple, la playlist sera composée de 50\% de genre '\texttt{rock}' et 20\% de l'album '\texttt{reversion}'.
\vfill
Document rédigé avec \LaTeX{}.
\end{document}
